%Overleaf (online website , pdfLaTeX)

\documentclass[]{article} 

\usepackage{amsmath}
\usepackage{hyperref}
\hypersetup{colorlinks=true, linkcolor=blue, urlcolor=blue, citecolor=blue}

\title{Hello World!}
\author{Your Name}
\date{January 1, 1831}

\pagenumbering{gobble} %To stop the page number from printing at the end of the page.

\begin{document}
	\maketitle
	\section{Getting Started}
	\textbf{Hello World!} Today I am learning \LaTeX. \LaTeX{} is a great program for writing math. I can write in line math such as $a^2+b^2=c^2$. I can also give equations their own space: 
	\begin{equation} 
		\gamma^2+\theta^2=\omega^2
	\end{equation}
	``Maxwell's equations'' are named for James Clark Maxwell and are as follow:
	\begin{align}             
		\vec{\nabla} \cdot \vec{E} \quad &=\quad\frac{\rho}{\epsilon_0} &&\text{Gauss's Law} \label{first}\\      
		\vec{\nabla} \cdot \vec{B} \quad &=\quad 0 &&\text{Gauss's Law for Magnetism} \label{second}\\
		\vec{\nabla} \times \vec{E} \quad &=\hspace{10pt}-\frac{\partial{\vec{B}}}{\partial{t}} &&\text{Faraday's Law of Induction} \label{third}\\ 
		\vec{\nabla} \times \vec{B} \quad &=\quad \mu_0\left( \epsilon_0\frac{\partial{\vec{E}}}{\partial{t}}+\vec{J}\right) &&\text{Ampere's Circuital Law} \label{fourth}
	\end{align}
	Equations \ref{first}, \ref{second}, \ref{third}, and \ref{fourth} are some of the most important in Physics.
	\section{What about Matrix Equations?}
	\begin{equation*}
		\begin{pmatrix}
			a_{11}&a_{12}&\dots&a_{1n}\\
			a_{21}&a_{22}&\dots&a_{2n}\\
			\vdots&\vdots&\ddots&\vdots\\
			a_{n1}&a_{n2}&\dots&a_{nn}
		\end{pmatrix}
		\begin{bmatrix}
			v_{1}\\
			v_{2}\\
			\vdots\\
			v_{n}
		\end{bmatrix}
		=
		\begin{matrix}
			w_{1}\\
			w_{2}\\
			\vdots\\
			w_{n}
		\end{matrix}
	\end{equation*}
\end{document}
